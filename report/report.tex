\documentclass[a4paper]{article}
\usepackage{iwslt15,amssymb,amsmath,epsfig}
\setcounter{page}{1}
\sloppy		% better line breaks
%\ninept
%SM below a registered trademark definition
\def\reg{{\rm\ooalign{\hfil
     \raise.07ex\hbox{\scriptsize R}\hfil\crcr\mathhexbox20D}}}

%% \newcommand{\reg}{\textsuperscript{\textcircled{\textsc r}}}

\title{Hand Gesture Recognition with Convolutional Neural Networks}

%%%%%%%%%%%%%%%%%%%%%%%%%%%%%%%%%%%%%%%%%%%%%%%%%%%%%%%%%%%%%%%%%%%%%%%%%%
%% If multiple authors, uncomment and edit the lines shown below.       %%
%% Note that each line must be emphasized {\em } by itself.             %%
%% (by Stephen Martucci, author of spconf.sty).                         %%
%%%%%%%%%%%%%%%%%%%%%%%%%%%%%%%%%%%%%%%%%%%%%%%%%%%%%%%%%%%%%%%%%%%%%%%%%%
% \makeatletter
% \def\name#1{\gdef\@name{#1\\}}
% \makeatother
% \name{{\em Firstname1 Lastname1, Firstname2 Lastname2, Firstname3 Lastname3,}\\
%      {\em Firstname4 Lastname4}}
%%%%%%%%%%%%%%% End of required multiple authors changes %%%%%%%%%%%%%%%%%

\address{Department of Speech Recognition and Machine Translation  \\
University of SomePlace, SomeCountry \\
{\small \tt firstname.lastname@iwslt.org}
}
%
\begin{document}
\maketitle
%
\begin{abstract}
This is a layout specification and template
definition for the paper of the IWSLT 2015 Conference. 
The format is essentially the one used for the IEEE ICASSP conferences.
\end{abstract}


%
\section{Introduction}
This template can be found on the conference 
website: \texttt{http://iwslt.org/}.
Please use either a MS-Word\reg\ or a \LaTeX\ format file
when preparing your submission. 

\section{Page layout and style}

Authors should should observe the following rules for page
layout. A highly  recommended way to meet these requirements
is to use a predefined template and check
details against the corresponding example file.

\subsection{First page}

The first page should have the paper title, author(s), and affiliation(s)
centered on the page across both columns. The remainder of the text 
must be in the two-column format, staying within the indicated image 
area. 

\subsubsection{Paper Title}
The paper title must be in boldface. All non-function words must be capitalized,
and all other words in the title must be lower case. The paper title is centered
across the top of the two columns on the first page as indicated above.

\subsubsection{Authors' Name(s)}
The authors' name(s) and affiliation(s) appear centered below the paper
title. If space permits, include a mailing address here. The templates indicate
the area where the title and author information should go. These items need not
be confined to the number of lines indicated; papers with multiple
authors and affiliations may require two or more lines. 
Note that the submission version of technical papers \emph{should be 
anonymized for review}. 

\subsubsection{Abstract}
Each paper must contain an abstract that appears at the beginning of the paper.

\subsection{Basic layout features}

\begin{itemize}
%\itemsep -1.3mm
\item Proceedings will be printed in A4 format. The layout is designed 
so that files, when printed in US Letter format, include all material 
but margins are not symmetric. 
Although this is not an absolute requirement, if at all possible,
{\bf PLEASE TRY TO MAKE YOUR SUBMISSION IN A4 FORMAT.}
\item Two columns are used except for the title part and possibly for large 
figures that need a full page width.
\item Left margin is 20 mm.
\item Column width is 80 mm.
\item Spacing between columns is 10 mm.
\item Top margin 25 mm (except first page 30 mm to title top).
\item Text height (without headers and footers) is maximum 235 mm.
\item Headers and footers must be left empty (they will be added for 
printing).
\item Check indentations and spacings by comparing to this 
example file (in pdf format).
\end{itemize}

\subsubsection{Headings}

Section headings are centered in boldface
with the first word capitalized and the rest of the heading in 
lower case. Sub-headings appear like major headings, except they 
start at the left margin in the column.
Sub-sub-headings appear like sub-headings, except they are in italics 
and not boldface. See the examples given in this 
file. No more than 3 levels of headings should be used.

\subsection{Text font}

Times or Times Roman font is used for the main text. Recommended 
font size is 9 points which is also the minimum allowed size.
Other font types may be used if needed for 
special purposes. While making the final PostScript file, 
remember to include all fonts!

\LaTeX\ users: DO NOT USE Computer Modern FONT FOR TEXT (Times is 
specified in the style file). If possible, make the final 
document using POSTSCRIPT FONTS.
This is necessary given that, for example, equations with 
non-ps Computer Modern are very hard to read on screen.

\subsection{Figures}

All figures must be centered on the column (or page, if the figure spans 
both columns).
Figure captions should follow each figure and have the format given in 
Fig.~\ref{spprod}.

Figures should preferably be line drawings. If they contain gray 
levels or colors, they should be checked to print well on a 
high-quality non-color laser printer.

\subsection{Tables}

An example of a table is shown as Table \ref{table1}. Somewhat 
different styles are allowed according to the type and purpose of the 
table. The caption text may be above or below the table.

\begin{table}
\caption{\label{table1} {\it This is an example of a table.}}
\vspace{2mm}
\centerline{
\begin{tabular}{|c|c|}
\hline
ratio & decibels \\
\hline  \hline
1/1 & 0 \\
2/1 & $\approx 6$ \\
3.16 & 10 \\
10/1 & 20 \\ 
1/10 & -20 \\
\hline
\end{tabular}}
\end{table}

\subsection{Equations}

Equations should be placed on separate lines and numbered. Examples 
of equations are given below.
Particularly,
%
%\vspace{-3mm}
\begin{equation}
x(t) = s(f_\omega(t))
\label{eq1}
\end{equation}
where \(f_\omega(t)\) is a special warping function
\begin{equation}
f_\omega(t)=\frac{1}{2\pi j}\oint_C \frac{\nu^{-1k}d\nu}
{(1-\beta\nu^{-1})(\nu^{-1}-\beta)}
\label{eq2}
\end{equation}
A residue theorem states that
\begin{equation}
\oint_C F(z)dz=2 \pi j \sum_k Res[F(z),p_k]
\label{eq3}
\end{equation}
Applying (\ref{eq3}) to (\ref{eq1}), 
it is straightforward to see that
\begin{equation}
1 + 1 = \pi
\label{eq4}
\end{equation}

Make sure to use \verb!\eqref! when refering to equation numbers.
Finally we have proven the secret theorem of all speech sciences (see
equation~\eqref{eq3} above).  No more math is needed to show how 
useful the result is! 

\begin{figure}[t]
\centerline{\epsfig{figure=figure,width=40mm}}
\caption{{\it Schematic diagram of speech production.}}  
\label{spprod}
\end{figure}

\subsection{Hyperlinks}

Hyperlinks can be included in your paper. Moreover, be aware that the paper
submission procedure includes the option of specifying a hyperlink for
additional information.  This hyperlink will be included in the CD-ROM.
Particularly pay attention to the possibility, from this single hyperlink, to
have further links to information such as other related documents, sound or
multimedia.

If you choose to use active hyperlinks in your paper, 
please make sure that they present no problems in printing to paper. 

\subsection{Page numbering}

Final page numbers will be added later to the document
electronically. 
{\em Please don't make any headers or footers!}.

\subsection{References}

The reference format is the standard for IEEE publications.
References should be numbered in order of appearance, 
for example \cite{ES1}, \cite{ES2}, and \cite{ES3}. 

\section{Experiments}
Please make sure to give all the necessary details regarding your experimental 
setting so as to ensure that your results could be reproduced by other teams. 

\section{Conclusions}

This paper has described a novel approach for doing wonderful stuff such as ...

\section{Acknowledgements}
The IWSLT 2015 organizing committee would like to thank the
organizing committees of INTERSPEECH 2004 for their
help and for kindly providing the template files.

%
\bibliographystyle{IEEEtran}
\begin{thebibliography}{10}
\bibitem[1]{ES1} Smith, J. O. and Abel, J. S., 
``Bark and {ERB} Bilinear Transforms'', 
IEEE Trans. Speech and Audio Proc., 7(6):697--708, 1999.  
\bibitem[2]{ES2} Lee, K.-F., Automatic Speech Recognition: 
The Development of the 
SPHINX SYSTEM, Kluwer Academic Publishers, Boston, 1989.
\bibitem[3]{ES3} Rudnicky, A. I., Polifroni, Thayer, E. H.,
 and Brennan, R. A.  
"Interactive problem solving with speech", J. Acoust. Soc. Amer., 
Vol. 84, 1988, p S213(A).
\end{thebibliography}
\end{document}

